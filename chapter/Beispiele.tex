\chapter{Example} \label{c:beispiele}

The chapter Examples (see \autoref{c:beispiele}) demonstrates the possible functions and\index{and} possibilities of this LaTeX document.


\section{Source code}

%TODO:なんというのだ?「次の」か?
Nachfolgend der \autoref{lst:helloworld}.

\begin{lstlisting}[caption={Hello World}, captionpos=b, label={lst:helloworld}]
/**
* The HelloWorldApp class implements an application that
* simply prints "Hello World!" to standard output.
*/
class HelloWorldApp {
	public static void main(String[] args) {
		System.out.println("Hello World!"); // Display the string.
	}
}
\end{lstlisting}



\section{Picture}

\begin{wrapfigure}{R}{0.5\textwidth}
	\centering
	\includegraphics[width=0.5\textwidth]{resources/images/example}
	\caption{sample picture {\cite{PEXELS2015}}}
\end{wrapfigure}

%Die rechts zu sehende Grafik demonstriert die Möglichkeiten des Paketes \glqq wrapfig\grqq . Grafiken innerhalb einer \glqq wrapfigure\grqq{} können entweder links oder rechts von Text umlaufen werden.
The graphic on the right demonstrates the possibilities of the package \glqq wrapfig\grqq . Graphics within a \glqq wrapfigure\grqq{} can be wrapped either left or right of text.

%Die nachfolgende \autoref{img:beispielbild} demonstriert die Darstellung\index{Darstellung} eines \glqq *.jpg\grqq{} Bildes innerhalb des Textes (beim Einfügen kann auf die Endung verzichtet werden, solange der Name einzigartig ist). Zusätzlich enthält dieses einen Untertitel der über das bereits verwendete Label verlinkt werden kann. Der Untertitel\index{Untertitel} erscheint im \gls{abbvz}.
The following \autoref{img:examplePicture} demonstrates the representation\index{Darstellung} of a \glqq *.jpg\grqq{} image within the text (when pasting, the extension can be omitted as long as the name is unique). Additionally, it contains a subtitle which can be linked to the already used label. The subtitle\index{Untertitel} appears in \gls{abbvz}.


\section{Text formatting etc}
This text contains a footnote\footnote{Footnotes are annotations that are extracted from the continuous text in the print layout to make the text easier to read.}

\subsection{Listen}
Lists can be created with bullet points as well as with numbers
\begin{itemize}
	\item a list with bullet points
	\item a further element
\end{itemize}

\begin{enumerate}
	\item a list with numbers
	\item a further element
\end{enumerate}

\subsection{Text Highlighting}
\begin{quote}
	The problem with internet quotes is that you can't always depend on their accuracy \par\raggedleft--- \textup{Abraham Lincoln, 1864}
\end{quote}

% in Lualatex 1.10.0 (TeX Live 2019/Debian), there need to be empty lines around the \epigraph{}{}
% https://tex.stackexchange.com/questions/505409/create-a-page-text-wide-text-box-in-twocolumn-document
Inspiring quotes can be inserted with epigraph

\epigraph{The problem with internet quotes is that you can't always depend on their accuracy}{Abraham Lincoln, 1864}

line breaks can only be written directly after text, otherwise the latex cannot be compiled.
\\ %これ


\begin{figure}[H]
	\centering
	\includegraphics[width=0.7\textwidth]{resources/images/example}
	\caption{sample picture {\cite{PEXELS2015}}}
	\label{img:examplePicture}
\end{figure}

\section{Table}

Nachfolgend \autoref{tbl:digitalCertificate}.

\begin{table}[H]
	\begin{center}
		\renewcommand{\arraystretch}{1.3}
		\begin{tabular}{|l|}
			\hline
			\textbf{owner:}\\
			Alice \\ \hline
			\textbf{Peer (creator):}\\
			Bob \\ \hline
			\textbf{Public key of the owner:}\\
			F2 D2 0E ED FA 4E 9E 0A F2 DD 23 8A 32 44 F3 E9 \\ \hline
			\textbf{validity:}\\
			2015-07-01 – 2016-06-30 \\ \hline
		\end{tabular}
	\end{center}
	\caption{digital certificate}
	\label{tbl:digitalCertificate}
\end{table}

\section{Long-Table}

%Die \glqq Long-Table\grqq kann über definierte Header und Footer über Seitenumbrüche hinweg angezeigt werden.
The \glqq Long-Table\grqq can be displayed over defined headers and footers across page breaks.


\begin{longtable}{|l|l|l|l|}
	\hline
	\multicolumn{1}{|c}{\textbf{Version}} & \multicolumn{1}{|c}{\textbf{Codename}} &
	\multicolumn{1}{|c}{\textbf{API}} &
	\multicolumn{1}{|c|}{\textbf{Verteilung}} \\ \hline
	\endfirsthead
	
	\multicolumn{4}{c}{... - Distribution of the Android versions (as of 2016-02-01)}\\ \hline
	\multicolumn{1}{|c}{\textbf{Version}} & \multicolumn{1}{|c}{\textbf{Codename}} &
	\multicolumn{1}{|c}{\textbf{API}} &
	\multicolumn{1}{|c|}{\textbf{Verteilung}} \\ \hline 
	\endhead
	
	\multicolumn{4}{c}{continued on next page}
	\endfoot
	
	\caption{Distribution of the Android versions (as of 2016-02-01)}
	\label{tab:androidverteilung}
	\endlastfoot
	
	2.2 & Froyo & 8 & 0.1\%\\ \hline
	2.3.3 - 2.3.7 & Gingerbread & 10 & 2.7\%\\ \hline
	4.0.3 - 4.0.4 & Ice Cream Sandwich & 15 & 2.5\%\\ \hline
	4.1.x & Jelly Bean & 16 & 8.8\%\\ \cline{1-1} \cline{3-4}
	4.2.x &  & 17 & 11.7\%\\ \cline{1-1} \cline{3-4}
	4.3 &  & 18 & 3.4\%\\ \hline
	4.4 & KitKat & 19 & 35.5\%\\ \hline
	5.0 & Lollipop & 21 & 17.0\%\\ \cline{1-1} \cline{3-4}
	5.1 &  & 22 & 17.1\%\\ \hline
	6.0 & Marshmallow & 23 & 1.2\%\\ \hline
\end{longtable}


\section{References}
Weil für die alte\index{alte} und die neue Rechtschreibung verschiedene Trennregeln\index{Trennregeln} gelten, sind Deutsch mit alter Rechtschreibung und Deutsch mit neuer Rechtschreibung zwei verschiedene Sprachen (\cite{Knappen2009}, S. 192).


\section{References (Online)}
Siehe Google.de \cite{Google2015}.


\section{Glossar}
The glossary contains the description of terms used for a better understanding towards the reader. Examples are: \gls{berlin}, \gls{outsourcing}, \gls{asp}, \gls{policy} and \gls{pcie}.


\section{Abkürzungsverzeichnis}
The list of abbreviations lists all abbreviations used. Some examples are \gls{sas}, \gls{cd}, \gls{lan} and \gls{iso}. The reuse only shows the abbreviation: \gls{sas}, \gls{cd}, \gls{lan} and\index{and} \gls{iso}.