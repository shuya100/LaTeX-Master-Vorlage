%Dokumenteinstellungen und Anpassungen
% Document class "scrbook" - Extended by the reference to the directories and text properties
\documentclass[chapterprefix=true, 12pt, a4paper, oneside, parskip=half, listof=totoc, bibliography=totoc, numbers=noendperiod]{scrbook}


% Adjustment of the page margins (standard bottom approx. 52mm with respect to approx. 4mm for the page number moved to the top right)
\usepackage[bottom=48mm,left=25mm,right=25mm]{geometry}

% Tweaks for scrbook
\usepackage{scrhack}

%Filler Text
\usepackage{blindtext}

% Permitted under other bridge captions
\usepackage{caption}

% List of Keywords
\usepackage{imakeidx}

% Compact lists
\usepackage{paralist}

% Better format and display quotes
\usepackage{epigraph}

% Glossary, keyword index (acronyms are listed separately)
\usepackage[toc, acronym]{glossaries} 

% Customization of header and footer
% influences the first page of the chapter
\usepackage[automark,headsepline]{scrlayer-scrpage}
\input{resources/styles/header_footer}

% Comment out for the reduction of the vertical distance of a new chapter
%\renewcommand*{\chapterheadstartvskip}{\vspace*{.25\baselineskip}}

% line spacing 1.5
\usepackage[onehalfspacing]{setspace}

% Improved representation of the letters to each other
\usepackage[stretch=10]{microtype}

% use English (e.g. table of contents etc.)
\usepackage[english]{babel}

% Support of umlauts and other special characters (UTF-8)
\usepackage{lmodern}
\usepackage[utf8]{luainputenc}
\usepackage[T1]{fontenc}

% Simpler quotes
\usepackage{epigraph}

% Support of H positioning (no automatic movement of inserted elements)
\usepackage{float} 

% Allows wrapping within tables
\usepackage{tabularx}

% Allows page breaks within tables
\usepackage{longtable}

% Allows to display source code with highlighting
\usepackage{listings}

% Definition of own colors when using a self-assigned name
\usepackage[table,xcdraw]{xcolor}

%Vektorgrafiken
\usepackage{tikz}

%Grafiken (wie jpg, png, etc.)
\usepackage{graphicx}

%Grafiken von Text umlaufen lassen
\usepackage{wrapfig}

%Ermöglicht Verknüpfungen innerhalb des Dokumentes (e.g. for PDF), Links werden durch "hidelink" nicht explizit hervorgehoben
\usepackage[hidelinks,german]{hyperref}

%Einbindung und Verwaltung von Literaturverzeichnissen
\usepackage{csquotes} %wird von biber benötigt
\usepackage[style=alphabetic, backend=biber, bibencoding=ascii]{biblatex}
\addbibresource{references/references.bib}
%Anpassung der Überschriften
\addtokomafont{disposition}{\rmfamily}

%Zusätzliche Farben
\definecolor{darkgreen}{RGB}{0,100,0}

%Umbenennungen
\renewcommand{\lstlistlistingname}{List of Source Codes}

%Pluszeichen in der Referenc beim zitieren ausblenden
\renewcommand*{\labelalphaothers}{}

%Anpassugen zur Quelltextdarstellung, kann bei Bedarf überschrieben werden (z.B. wenn unterschiedliche Sprachen zum Einsatz kommen)
\renewcommand{\lstlistingname}{Codeauszug}
\lstset{
	language=Java,
	numbers=left,
	columns=fullflexible,
	aboveskip=5pt,
	belowskip=10pt,
	basicstyle=\small\ttfamily,
	backgroundcolor=\color{black!5},
	commentstyle=\color{darkgreen},
	keywordstyle=\color{blue},
	stringstyle=\color{gray},
	showspaces=false,
	showstringspaces=false,
	showtabs=false,
	xleftmargin=16pt,
	xrightmargin=0pt,
	framesep=5pt,
	framerule=3pt,
	frame=leftline,
	rulecolor=\color{green},
	tabsize=2,
	breaklines=true,
	breakatwhitespace=true,
	prebreak={\mbox{$\hookleftarrow$}}
}

%Anpassungen für das Abkürzungsverzeichnis
\newglossarystyle{dottedlocations}{%
	\renewcommand*{\glossaryentryfield}[5]{%
		\item[\glsentryitem{##1}\glstarget{##1}{##2}] \emph{##3}%
		\unskip\leaders\hbox to 2.9mm{\hss.}\hfill##5}%
	\renewcommand*{\glsgroupskip}{}%
}

%Titelformen - gewünschtes Layout einkommentieren

%%Graduation
\makeatletter

\newcommand*{\gradeType}[1]{\gdef\@gradeType{#1}}
\newcommand*{\firstExaminer}[1]{\gdef\@firstExaminer{#1}}
\newcommand*{\secondExaminer}[1]{\gdef\@secondExaminer{#1}}
\newcommand*{\matrikelnr}[1]{\gdef\@matrikelnr{#1}}
\newcommand*{\submitDate}[1]{\gdef\@submitDate{#1}}

\renewcommand*{\maketitle}{
	\begin{titlepage}
		\newgeometry{left=2.5cm,right=2.5cm,top=9.0cm,bottom=2.5cm}
		\begin{center}
			\vfill
			{\Large \@title\par}
			\vskip 0.5cm
			{\large \bfseries Bacherlor's/Master's Thesis\par}
			\vskip 0.5cm
						% {\large for the purpose of obtaining the academic degree \\ 
						{\large in partial fulfillment of the requirements \\ 
						\large for the degree of					
						\bfseries \@gradeType}
						\vskip 0.5cm
			{\large in}
			\vskip 0.5cm
			
			%TODO: for が2回も使われているのが鬱陶しい。
			{\large Hochschule für Technik und Wirtschaft Berlin
	\\ (HTW Berlin - University of Applied Sciences)
	\\ Fachbereich 4 Informatik, Kommunikation und Wirtschaft\\ Internationaler Studiengang Medieninformatik}
			\vfill
			\begin{flushleft}
				\begin{tabular}[t]{rl}
					1. Supervisor: &\@firstExaminer\\
					2. Supervisor: & \@secondExaminer\\
					\\
					Author : &\@author\\
					Matriculation Number: & \@matrikelnr\\
					Date of Submission: & \@submitDate
				\end{tabular}
			\end{flushleft}
		\end{center}
		\restoregeometry
	\end{titlepage}
}
\makeatother
\gradeType{Master of Science (M.Sc.)}
\secondExaminer{John Doe}

%Research paper
%\include{titles/research_papger}
%\subTitle{Ein optionaler Untertitel der Arbeit}
%\researchPart{A}

%Angaben zur Arbeit und dem Author (von beiden Layouts genutzt)
\title{This is the title of the thesis which can extend over several lines}
\author{John Doe}
\matrikelnr{s0000000}
\submitDate{February 13, 2022}
\firstExaminer{John Doe}

%Verzeichnisse generieren
\makeglossaries
\loadglsentries{references/glossary_acronyms.tex}
\setacronymstyle{long-short}

\makeindex[columns=2, title=Stichwortverzeichnis, options= -s resources/styles/indexstyle.ist, intoc]
\indexsetup{level=\chapter*,toclevel=chapter}

%Start des Inhalts
\begin{document}

%Notwendiger Workaround
\pagenumbering{alph}

%Deckblatt erzeugen
\maketitle

\pagenumbering{Roman}

\chapter*{Preamble}
\blindtext \clearpage
\include{chapter/Abstract} \clearpage

%Inhaltsverzeichnis
\tableofcontents \newpage

%Hauptteil
\pagenumbering{arabic}
\chapter{Introduction}
\section{Filler Text}
\Blindtext[2][3] 
\blinditemize \clearpage
\chapter{Beispiele} \label{c:beispiele}

Im Kapitel Beispiele (siehe \autoref{c:beispiele}) werden die möglichen Funktionen und\index{und} Möglichkeiten dies LaTeX-Dokuments demonstriert.

\section{Source Code}

Nachfolgend der \autoref{lst:helloworld}.

\begin{lstlisting}[caption={Hello World}, captionpos=b, label={lst:helloworld}]
/**
* The HelloWorldApp class implements an application that
* simply prints "Hello World!" to standard output.
*/
class HelloWorldApp {
	public static void main(String[] args) {
		System.out.println("Hello World!"); // Display the string.
	}
}
\end{lstlisting}

\section{Picture}

\begin{wrapfigure}{R}{0.5\textwidth}
	\centering
	\includegraphics[width=0.5\textwidth]{resources/images/example}
	\caption{Beispielbild {\cite{PEXELS2015}}}
\end{wrapfigure}

Die rechts zu sehende Grafik demonstriert die Möglichkeiten des Paketes \glqq wrapfig\grqq . Grafiken innerhalb einer \glqq wrapfigure\grqq{} können entweder links oder rechts von Text umlaufen werden.

Die nachfolgende \autoref{img:beispielbild} demonstriert die Darstellung\index{Darstellung} eines \glqq *.jpg\grqq{} Bildes innerhalb des Textes (beim Einfügen kann auf die Endung verzichtet werden, solange der Name einzigartig ist). Zusätzlich enthält dieses einen Untertitel der über das bereits verwendete Label verlinkt werden kann. Der Untertitel\index{Untertitel} erscheint im \gls{abbvz}.

\section{Text Formatierungen und sonstiges}
Dieser Text enthält eine Fußnote\footnote{Fußnoten sind Anmerkungen, die im Druck-Layout aus dem Fließtext ausgelagert werden, um den Text flüssig lesbar zu gestalten.}.

\subsection{Listen}
Listen könne sowohl mit Bullet points als auch mit Zahlen erstellt werden
\begin{itemize}
	\item Eine Liste mit Bullet points
	\item Ein weiteres Element
\end{itemize}

\begin{enumerate}
	\item Eine Liste mit Zahlen
	\item Ein weiteres Element
\end{enumerate}

\subsection{Text Hervorhebungen}
\begin{quote}
	The problem with internet quotes is that you can't always depend on their accuracy \par\raggedleft--- \textup{Abraham Lincoln, 1864}
\end{quote}

"Inspirierende Zitate können mit epigraph eingefügt werden
%\epigraph{The problem with internet quotes is that you can't always depend on their accuracy}{Abraham Lincoln, 1864}

Seitenumbrüche können nur direkt nach Text geschrieben werden, sonst lässt sich das Latex nicht mehr compilieren.
\\

\begin{figure}[H]
	\centering
	\includegraphics[width=0.7\textwidth]{resources/images/example}
	\caption{Beispielbild {\cite{PEXELS2015}}}
	\label{img:beispielbild}
\end{figure}

\section{Tabelle}

Nachfolgend \autoref{tbl:DigitalesZertifikat}.

\begin{table}[H]
	\begin{center}
		\renewcommand{\arraystretch}{1.3}
		\begin{tabular}{|l|}
			\hline
			\textbf{Inhaber:}\\
			Alice \\ \hline
			\textbf{Peer (Ersteller):}\\
			Bob \\ \hline
			\textbf{Öffentlicher Schlüssel des Inhabers:}\\
			F2 D2 0E ED FA 4E 9E 0A F2 DD 23 8A 32 44 F3 E9 \\ \hline
			\textbf{Gültigkeit:}\\
			2015-07-01 – 2016-06-30 \\ \hline
		\end{tabular}
	\end{center}
	\caption{Digitales Zertifikat}
	\label{tbl:DigitalesZertifikat}
\end{table}

\section{Long-Table}

Die \glqq Long-Table\grqq kann über definierte Header und Footer über Seitenumbrüche hinweg angezeigt werden.

\begin{longtable}{|l|l|l|l|}
	\hline
	\multicolumn{1}{|c}{\textbf{Version}} & \multicolumn{1}{|c}{\textbf{Codename}} &
	\multicolumn{1}{|c}{\textbf{API}} &
	\multicolumn{1}{|c|}{\textbf{Verteilung}} \\ \hline
	\endfirsthead
	
	\multicolumn{4}{c}{Fortsetzung - Verteilung der Androidversionen (Stand 01.02.2016)}\\ \hline
	\multicolumn{1}{|c}{\textbf{Version}} & \multicolumn{1}{|c}{\textbf{Codename}} &
	\multicolumn{1}{|c}{\textbf{API}} &
	\multicolumn{1}{|c|}{\textbf{Verteilung}} \\ \hline 
	\endhead
	
	\multicolumn{4}{c}{Fortsetzung auf nachfolgender Seite}
	\endfoot
	
	\caption{Verteilung der Androidversionen (Stand: 01.02.2016)}
	\label{tab:androidverteilung}
	\endlastfoot
	
	2.2 & Froyo & 8 & 0.1\%\\ \hline
	2.3.3 - 2.3.7 & Gingerbread & 10 & 2.7\%\\ \hline
	4.0.3 - 4.0.4 & Ice Cream Sandwich & 15 & 2.5\%\\ \hline
	4.1.x & Jelly Bean & 16 & 8.8\%\\ \cline{1-1} \cline{3-4}
	4.2.x &  & 17 & 11.7\%\\ \cline{1-1} \cline{3-4}
	4.3 &  & 18 & 3.4\%\\ \hline
	4.4 & KitKat & 19 & 35.5\%\\ \hline
	5.0 & Lollipop & 21 & 17.0\%\\ \cline{1-1} \cline{3-4}
	5.1 &  & 22 & 17.1\%\\ \hline
	6.0 & Marshmallow & 23 & 1.2\%\\ \hline
\end{longtable}

\section{Literaturverweis}

Weil für die alte\index{alte} und die neue Rechtschreibung verschiedene Trennregeln\index{Trennregeln} gelten, sind Deutsch mit alter Rechtschreibung und Deutsch mit neuer Rechtschreibung zwei verschiedene Sprachen (\cite{Knappen2009}, S. 192).

\section{Onlineverweise}

Siehe Google.de \cite{Google2015}.

\section{Glossar}
Der Glossar enthält die Beschreibung verwendeter Begriffe für das bessere Verständnis gegenüber dem Leser. Beispiele sind: \gls{berlin}, \gls{outsourcing}, \gls{asp}, \gls{policy} und \gls{pcie}.

\section{Abkürzungsverzeichnis}
Das Abkürzungsverzeichnis listet alle verwendeten Abkürzungen auf. Einige Beispiele sind \gls{sas}, \gls{cd}, \gls{lan} und \gls{iso}. Die erneute Verwendung zeigt nur noch die Abkürzung: \gls{sas}, \gls{cd}, \gls{lan} und\index{und} \gls{iso}. \clearpage

%Anhang
\pagenumbering{Alph}

%Abbildungsverzeichnis
\listoffigures \clearpage
%Tabellenverzeichnis
\listoftables \clearpage
%Quelltextverzeichnis
\lstlistoflistings \clearpage
%Stichwortverzeichnis
\printindex \clearpage
%Glossar
\printglossary[title={Glossar}] \clearpage
%Abkürzungsverzeichnis
\printglossary[style=dottedlocations,type=\acronymtype,title={Abkürzungsverzeichnis}] \clearpage

%Literaturverzeichnisse (getrennt nach Stichwort)
\printbibliography[heading=bibintoc, keyword={book}, title={Literaturverzeichnis}]\clearpage
\printbibliography[heading=bibintoc, keyword={online}, title={Onlinequellen}]\clearpage
\printbibliography[heading=bibintoc, keyword={image}, title={Bildquellen}]\clearpage

% Anhang
\appendix

\chapter{}
\addcontentsline{toc}{chapter}{Appendix}

\section{Diagram}

\section{Table}

\section{Screenshot}

\section{Graph}

% Eigenständigkeitserklärung
\addchap{Declaration}

I hereby declare and confirm that this thesis is entirely the result of my own original work. Where other sources of information have been used, they have been indicated as such and Properly acknowledged. I further declare that this or similar work has not been submitted for credit elsewhere.


\vskip 1cm

Berlin, February 2?, 2021

\vskip 1.5cm

John Doe

\end{document}